\begin{secao}{Atitude, bixes!}

Na USP, os alunos têm a liberdade e apoio de se organizarem
para montar grupos de debates, ciclos de palestras, grupos
de desenvolvimento e até mesmo grupos para jogarem alguma coisa (como
RPG, Magic, Yu-Gi-Oh ou algum esporte).
Portanto, caso vocês tenham algum projeto em mente, não hesitem
em se organizarem com seus amigos e se informarem em como avançar com essa
ideia. Lembre-se de que seus veteranes estão aí para te aconselhar e tirar
eventuais dúvidas.

É possível também juntar-se com alguns amigos e formar grupos de
estudo, seja para alguma matéria com a qual vocês tenham dificuldade, para
discutir aquele EP/lista de exercícios que ninguém está conseguindo
fazer ou simplesmente estudar algum tópico de interesse mútuo.

A seguir, temos (majoritariamente) alguns dos exemplos dos grupos que foram 
direta ou indiretamente criados por alunos do nosso Instituto!

% RDs --------------------------------------------------------------------------
\begin{subsecao}{RDs}

Antes de mais nada, RD significa Representante Discente. O RD é um aluno que
representa nossos interesses frente aos diversos conselhos e comissões
existentes, sendo um forte elo de ligação entre professores e alunos. O RD
ajuda a tomar decisões que impactam todo o IME, como autorização para festas,
mudanças no currículo, aumento de vagas na FUVEST, quantidade de bolsas,
reformas, mudança no corpo docente (às vezes lutamos para tirar algum
professor), enfim, coisas desse tipo e muitas mais.

Acho que já deu para perceber o quanto é importante ter um estudante em cada um
desses conselhos. Infelizmente, não costumamos preencher todas as vagas. Isso se
deve ao desinteresse de alguns ou falta de tempo da maioria de seus veteranes.
E vocês são quem tem mais tempo para fazer as coisas funcionarem aqui, já que
ainda não sabem o que é rec, DP, trabalho, estágio etc. Portanto, se quiserem
fazer alguma coisa pelo lugar onde vocês, bixes, vão estudar, está aí
uma dica.

Como RDs vocês poderão entender melhor o funcionamento do Instituto e ajudar no
processo de melhorá-lo. Vocês também poderão melhorar o relacionamento entre
estudantes, professores e servidores e entender como os professores pensam.
As eleições são organizadas pela direção do instituto no final do segundo
semestre e o mandato é de um ano.

Aqui vai um breve resumo do que mais ou menos acontece em cada um dos colegiados
nos quais temos direito a representante(s):

No IME, temos 24 cargos de RD com 34 vagas no total, sendo 17 reservadas
para graduação, 14 para pós e 3 livres. Todos têm direito a um suplente.
Caso o RD não possa ir a alguma reunião ou, por algum motivo da vida, tenha
que abandonar o cargo, o suplente assume em seu lugar e o cargo não fica
sem representante.

Existem diferentes níveis de hierarquia na administração.

{\bf As CoCs,
Comissões Coordenadoras de Curso (Lic, Pura, Estatística, Aplicada, BMAC e
Computação)} são as mais próximas dos alunos. Temos um cargo de aluno em cada
comissão. São comissões pequenas, que tratam dos problemas internos de cada
curso: mudança de currículo, requerimentos, optativas, etc. São subordinadas
à CG e ao conselho do relativo departamento. Analogamente, temos um cargo em cada
Comissão Coordenadora de Programa (de Pós).

{\bf Os Conselhos de Departamento (MAT, MAE, MAC e MAP)} têm uma dinâmica um
pouco diferente das CoCs: são mais formais. Cada conselho se reúne (quase)
mensalmente e são formados (em geral) por mais pessoas, sendo que existem
regras sobre participação dos diferentes níveis hierárquicos de
professores (Titular, Associado, Doutor e Assistente). Nesses conselhos, além
de aprovar algumas das decisões das Comissões Coordenadoras de Curso e de
Programa (pós) e distribuição de carga didática, são discutidos re-oferecimento
de curso, revisão de prova, supervisão das atividades dos docentes,
afastamentos (temporários ou não), contratação de professores e muitas outras
coisas.
Os Conselhos de Departamento são subordinados à Congregação e ao CTA.

{\bf A Comissão de Graduação (CG)}, basicamente, avalia requerimentos,
mudança/criação de cursos e jubilamentos. Analogamente, existe a Comissão de
Pós-Graduação (CPG). Ambas são subordinadas à Congregação.

{\bf A Comissão de Cultura e Extensão (CCEx)} quase nunca tem reunião. Cuida
das atividades de extensão: Matemateca, CAEM etc.

Também há comissões mais específicas, como a comissão de estágio, a comissão de
pesquisa (do doutorado) e o Centro de Competência em Software Livre (CCSL), da
computação.

Os dois conselhos mais importantes são o CTA e a Congregação, ambos presididos
pelo Diretor.

{\bf O Conselho Técnico e Administrativo (CTA)} cuida de todas as questões não
acadêmicas: Orçamento, reformas, avaliação dos funcionários, Xerox, etc. É
formado pelos quatro chefes de departamento, diretor, vice-diretor, um
representante dos funcionários e um RD.

{\bf A Congregação} é o órgão máximo do Instituto. Inclui muitos professores, a
maioria titular. São 3 RDs de graduação e 2 de Pós. Basicamente,
nesse órgão, são rediscutidas e aprovadas (ou não) muitas das decisões
dos órgãos subordinados. Os membros da Congregação têm voto na eleição para
Reitor e Vice-Reitor.

Bom, caso não tenha ficado claro desde o começo desse texto, percebam que é
muito importante ter um aluno em cada um desses conselhos. Se estiverem tendo
problemas com professores, requerimentos etc., ou simplesmente quiserem saber
o que anda acontecendo, procurem o RD certo pra conversar. Perguntem,
participem, votem e façam o IME um lugar melhor.

Sobre a eleição dos RDs: A eleição oficial para os RDs acontece no final do ano
(então fiquem atentos!) e é organizada pela diretoria do instituto. Os
interessados devem preencher um formulário de inscrição e levar até a
Assistência Acadêmica do IME, que vai organizar todos os inscritos e abrir um
processo de eleição online (em que todo IMEano pode votar). Quando a eleição
estiver se aproximando, vocês receberão (vários) e-mails com os documentos
necessários, prazos e links de votação.

%REFTIME
O resultado da eleição anterior com os RDs de 2019 pode ser encontrada no site
do IME:
\url{www.ime.usp.br/eleicoes-estatutarias}

\end{subsecao}


% Rede Linux -------------------------------------------------------------------
\begin{subsecao}{Rede Linux}

\figurapequenainline{rede_linux}

\begin{subsubsecao}{Introdução}

A Rede Linux é uma rede de computadores, administrada por alunos do IME e
que fornece diversos serviços para todos os alunos do IME.
Ela disponibiliza:

%FIXME
\vspace{-1em}

\begin{itemize}
\item 2 salas de computadores (no bloco A) com todo\footnote{ Se um programa
estiver faltando, mande um email pra admin@linux.ime.usp.br pedindo-o.} tipo de
programa necessário para suas atividades acadêmicas (com pelo menos uma que fica
aberta 24 horas por dia, 7 dias por semana\footnote{ Mas talvez vocês não
consigam entrar no bloco A depois da meia noite, que é quando a portaria
fecha.});
\item Uma página na internet para cada aluno;
\item Um \textit{e-mail} para cada aluno;
\item Espaço para você guardar seus arquivos;
\item Acesso remoto via ssh (linux.ime.usp.br);
\item Impressoras;
\item Admins dispostos e capazes, para o caso de algum usuário ter alguma boa
ideia para adicionar a esta lista;
\end{itemize}
\end{subsubsecao}

\begin{subsubsecao}{O Linux}

A rede utiliza em todos os seus computadores um sistema operacional chamado
Linux. Esse é um sistema desenvolvido de forma colaborativa pelos usuários
e empresas interessados nele (se quiser saber mais a respeito, pesquise
por ``software livre''!).

O Linux não é um sistema mais difícil de usar do que o Windows. É apenas
diferente em alguns aspectos. Além de tudo, existem cursos de Linux que são
organizados pelos alunos do IME. Os admins costumam promover esses cursos.
Fiquem atentos aos emails!

Não se deixem intimidar pelo sistema. Se vocês se derem ao trabalho de
aprender a utilizá-lo bem, verão que ele é bastante flexível, e até mesmo
interessante (tanto quanto um sistema operacional pode ser =P).

\end{subsubsecao}

\begin{subsubsecao}{Os admins}

Os admins são alunos do bacharelado em ciência da computação (vulgo BCC) que
são responsáveis por administrar a rede. Entre outras coisas, isso quer dizer
manter os computadores funcionando, ajudar os alunos a usar a rede (com
cursos\footnote{ Fiquem atentos aos emails!!!} e resolvendo dúvidas nos
horários de plantão\footnote{ Na página da rede, estão os horários de todos os
admins.}) e também implementar coisas novas na rede (aceitamos sugestões!)

Os admins são escolhidos por um treinamento que acontece de dois em dois anos,
em todo ano par. Mais informações serão divulgadas quando este estiver próximo
a ocorrer.

\end{subsubsecao}
\begin{subsubsecao}{Como criar uma conta?}

Basta passar na Admin, na sala 125 do bloco A (como vocês são bixes: bloco A é o da
biblioteca, bloco B aquele que tem muitas salas de aula e que vocês vão passar boa
parte da vida de vocês). Contatos:

%FIXME
\vspace{-1em}

\begin{description}
\item [e-mail:] admin@linux.ime.usp.br
\item [Página:] \url{www.linux.ime.usp.br}
\item [Sala:] 125, bloco A
\end{description}

%FIXME
\vspace{-.5em}

\end{subsubsecao}

\end{subsecao}


% FLUSP ------------------------------------------------------------------------
\begin{subsecao}{FLUSP}

\figurapequenainline{flusp}

Você já ouviu falar sobre Linux? E GCC? Talvez Gimp, qbittorrent, VLC ou Firefox?
Todos estes projetos são chamados de FLOSS: Free Libre Open Source Software, e
têm em comum a liberdade de código e conhecimento. Isto quer dizer que qualquer
um pode ler, usar e contribuir para estes projetos, seja com código ou arte!

O FLUSP: FLOSS@USP é um grupo de extensão com o objetivo de reunir alunos de
graduação e pós-graduação interessados em contribuir para projetos FLOSS.
Atualmente, temos contribuidores no Kernel Linux, no compilador GCC, no controlador 
de versão git, no projeto Caninos Loucos e muitos outros. Em 2019, o FLUSP 
foi responsável por aproximadamente 20\% das contribuições para o Kernel Linux no 
subsistema Industrial Input/Output. Devido a nossas contribuições para os drivers 
da Analog Devices, a empresa doou duas placas de testes ao grupo. Além delas, temos 
uma placa Labrador doada pelo projeto Caninos Loucos.

Assim como na comunidade FLOSS, nós encorajamos a liberdade no FLUSP. Seja para
contribuir com um projeto existente ou compartilhar um projeto pessoal para
outros contribuirem, aqui você encontra espaço!

\begin{description}
  \item[Facebook:] \url{facebook.com/flusp}
  \item[Telegram:] \url{http://tiny.cc/flusp}
  \item[IRC:] Servidor \texttt{irc.freenode.net}, canal \texttt{\#ccsl-usp}
  \item[Site:] \url{https://flusp.ime.usp.br}
  \item[Lista de email:] \texttt{flusp@googlegroups.com}
  \item[GitLab:] \url{https://gitlab.com/flusp}
\end{description}

\end{subsecao}


% IME Júnior -------------------------------------------------------------------
\input{ime_jr.tex}

% MaratonUSP -------------------------------------------------------------------
\begin{subsecao}{MaratonUSP}

\figurapequenainlineapertada{maratonusp}

O MaratonUSP é um grupo de estudos de programação competitiva. O principal evento é a Maratona de Programação, uma competição anual organizada pela Sociedade Brasileira de Computação (SBC). A competição é restrita a universitários que se organizam em times de três pessoas para resolver os mais variados problemas em forma de código. Apesar da coletividade, há uma dificuldade a mais: cada equipe dispõe de apenas um computador.

A Maratona é dividida em várias fases. Primeiro, há uma seleção interna dos times que representarão a USP na competição. Uma das vagas é reservada para um time de bixes! Depois, há uma fase regional com outras universidades da capital paulista, onde a USP tem um histórico de vitórias absoluto.

As coisas são consideravelmente mais difíceis na fase nacional, conhecida como Final Brasileira. As melhores universidades do Brasil se reúnem em algum canto do país para disputar as vagas brasileiras para a fase mundial. Exemplos recentes são Belo Horizonte, Foz do Iguaçu e Salvador. A hospedagem, comida e transporte são todos cobertos pela organização e pela universidade. Viagem de graça! Os melhores times brasileiros vão para a Final Mundial, sempre num país diferente. Nós estivemos recentemente no Marrocos, na Tailândia e na China. Em hotéis 5 estrelas! De graça!

Grandes empresas têm muito interesse nos alunos que participam da Maratona. É muito comum encontrar ex-maratonistas em empresas como Google, Facebook, etc. O pensamento abstrato e o trabalho em equipe desenvolvidos na Maratona são características muito bem avaliadas pelos recrutadores.

O ano começa com o bixeCamp, uma série de aulas focadas em ensinar o básico de programação e algoritmos. As aulas são no início dos treinos, que ocorrem no CEC (do lado da Seção de Alunos) à partir das 14h às sextas-feiras. O treino acaba às 19h, mas não é necessário ficar até o final. Alunos já experientes em programação são encorajados a comparecer também. Te esperamos lá!

\begin{description}
\item [Facebook:] \url{facebook.com/MaratonUSP}
\item[Site:] \url{www.ime.usp.br/~maratona}
\end{description}

\end{subsecao}


% USPCodeLab -------------------------------------------------------------------
\input{usp_code_lab.tex}

% USPGameDev: Pesquisa e Desenvolvimentos de Jogos na USP ----------------------
\begin{subsecao}{USPGameDev: Pesquisa e Desenvolvimentos de Jogos na USP}

\figurapequenainline{uspgamedev}

Valve. Blizzard. Rockstar. Nintendo. USPGameDev. O que esses nomes têm em comum?
São nomes de grupos de desenvolvedores de jogos. E um deles tem sua sede na USP.

Constituído primariamente, mas não exclusivamente, de alunos da USP de diversas
áreas (computação, matemática, design, música, letras, \textit{etc}.) o
USPGameDev (UGD) foi criado em 2009  e já lançou dezenas de jogos (um deles no 
Steam!\footnote{\texttt{https://store.steampowered.com/app/827940/Marvellous\_Inc/}}) 
e até mesmo o seu próprio \textit{kit} de desenvolvimento. Trabalhamos com jogos 
digitais e analógicos (video-games e jogos de tabuleiro, por exemplo). Vale 
ressaltar que adotamos a filosofia de software livre (\textit{``livre'' de 
``liberdade'', não necessariamente grátis}).

Queremos aprender e ensinar desenvolvimento de jogos como a atividade 
multifacetada que ela é. Você, que acabou de ingressar, também pode criar o seu 
próprio jogo dentro do USPGameDev, além de participar dos muitos projetos que já 
estão em andamento. Tanto que alguns dos nossos últimos jogos foram produtos de 
grupos de ingressantes (com muito tempo livre ou em disciplinas que aceitavam 
jogos como trabalhos). 

Não é necessário conhecimento prévio algum! Todos nós começamos a vida sem saber 
programar, desenvolver, projetar, desenhar, \textit{etc}. Isso porque não somos 
uma \textit{empresa} de jogos, mas um grupo de estudos. E, portanto, buscamos 
aprender o que não sabemos e ensinar o que sabemos.
Justamente por isso, o UGD também oferece cursos e \textit{workshops} livremente 
para a comunidade USP sobre diversos assuntos envolvendo desenvolvimento de 
jogos. Fiquem de olho! Além disso, participamos de \textbf{game jams} (ou 
hackathons): eventos regionais e internacionais onde temos de 24 a 72 horas para 
fazer um jogo com base em um tema que só é revelado na hora!

Interessados? Acessem nosso muito bem desenvolvido \textit{site} e deem uma 
conferida na página de downloads! 

\textbf{Para participar, basta entrar em contato (via email ou o que seja) 
conosco e marcar um dia para conversarmos}. Ser um membro não é lá muito formal, 
a gente bate um papo e vê o serial legal para você fazer. O grupo é horizontal e 
cada um escolhe quanto participa.

\begin{description}
  \item[Site:] \url{http://uspgamedev.org}
  \item[Fórum:] \url{forum.uspgamedev.org}
  \item[E-mail:] contato@uspgamedev.org
  \item[Facebook:] \url{facebook.com/UspGameDev}
\end{description}

\end{subsecao}


% IMEsec -----------------------------------------------------------------------
\begin{subsecao}{IMEsec}

\figurapequenainline{imesec}

O IMEsec é um grupo de extensão focado em aprender, estudar e se divertir com a
segurança da informação. Sem que a maioria das pessoas se dê conta, este nicho
está presente no cotidiano. Mandar uma mensagem no WhatsApp, navegar pela web,
entrar no Facebook: exemplos de ações simples do dia-a-dia que necessitam ser
feitas de maneira segura, visando à privacidade do usuário.

Nosso grupo, formado majoritariamente por alunos do IME-USP no início de 2017,
busca entender melhor este universo e expandí-lo no ambiente universitário. O
foco tem sido amplo; desde resolução de desafios on-line (que são muito
divertidos), participação em competições (sim, competições — \textit{capture the
flag} — muito, muito legais), apresentação de palestras e até desenvolvimento de
projetos que possam beneficiar a população. Todos são bem-vindos; basta ter
interesse pelo assunto.

As experiências decorrentes de nossas atividades ajudaram e ajudam nos estudos,
além de agregarem valor à graduação. Por meio desses, mergulhamos não só em
computação mas também em matemática, estatística e até outras áreas bem
inusitadas.

Fiquem de olho na nossa página do Facebook e do Telegram para saber mais sobre
as reuniões semanais, além de fatos interessantes (ou simplesmente engraçados)
sobre o vasto mundo da segurança. Participem!

\begin{description}
  \item[Facebook:] \url{facebook.com/imesec}
  \item[Telegram:] \url{tiny.cc/imesec-telegram}
  \item[Site:] \url{imesec.ime.usp.br/}
\end{description}

\end{subsecao}


% Hardware Livre ---------------------------------------------------------------
\input{hardware_livre.tex}

% Tecs: Computação Social ------------------------------------------------------

\begin{subsecao}{Tecs}

\figurapequenainline{tecs}

O Tecs é um grupo de extensão que iniciou suas atividades no segundo semestre de
2017, e é focado no impacto social da computação e da tecnologia. 

Buscamos construir uma rede de estudantes interessados em melhorar a comunidade
local e a sociedade e criar oportunidades para que eles possam engajar-se em
projetos com esse fim; unir esforços para formar uma sociedade e profissionais
éticos e conscientes sobre o uso da tecnologia; promover a educação tecnológica
igualitária da população por meio de cursos, oficinas e ações promovidas pelo
grupo; e estimular alunos a usarem a tecnologia para solucionar problemas da
comunidade. 

Em termos gerais, pretendemos que os estudantes entendam como a tecnologia pode
ser utilizada para o bem coletivo, e utilizem esse conhecimento na prática, por
meio de colaborações com a comunidade local, os serviços públicos, as
organizações não-governamentais e as sem fins lucrativos. 

Se você tem interesse em promover o ensino de computação, em debater questões
éticas e sociais no contexto tecnológico ou em desenvolver aplicativos, sites ou
sistemas em parcerias com projetos sociais, entre em contato conosco e participe
do grupo!

\vspace{-1em}
\begin{description}
  \item[Site:] \url{www.ime.usp.br/~tecs}
  \item[Facebook:] \url{www.facebook.com/tecs.usp}
  \item[Telegram:] \url{https://t.me/tecsusp}
\end{description}

\end{subsecao}


%FIXME GAMBIARRA para não quebrar página num lugar zuado
\pagebreak

% Diversime --------------------------------------------------------------------
\input{diversime.tex}

% Existimos! -------------------------------------------------------------------
\begin{subsecao}{$\exists$xistimos!}

\figurapequenainlineflexivel{existimos}{30pt}


$\exists$xistimos! surgiu em 2014 com a proposta de criar um espaço de
confiança entre as alunas do IME, onde cada uma possa ter a liberdade e
segurança para discutir suas vivências e propostas.


O desconforto com a forma como as mulheres são tratadas no IME e nas ciências
exatas em geral fez com que nos juntássemos para discutir como as questões de
gênero se manifestam no instituto e quais são as formas de agir para evitar
situações desconfortáveis ou preconceituosas.

Desde então promovemos uma série de eventos e intervenções para que tal debate
atinja toda a comunidade imeana, além de realizarmos reuniões periódicas apenas
com mulheres para que possamos conversar e nos ajudar, em qualquer situação,
mas em especial naquelas onde possamos ser vítimas ou testemunhas de  
preconceito e machismo. Todas vocês estão convidadas a participar das nossas  
reuniões, alunas de outros institutos são muito bem vindas $<$3.

Para falar conosco ou participar do grupo basta enviar um email ou entrar em
contato pelo facebook:


\begin{description}

\item[E-mail:] 3xistimos@gmail.com ou existimos@google.groups.com
\item[Facebook:] \url{https://www.fb.com/3xistimos/}
\item[Para denúncias ou relatos anônimos acesse:] \url{http://bit.ly/existimos}

\end{description}


\end{subsecao}



%FIXME GAMBIARRA para não quebrar página num lugar zuado
\pagebreak

% Comissão de Acolhimento da Mulher! -------------------------------------------
\input{cam.tex}

%FIXME GAMBIARRA para não quebrar página num lugar zuado
\pagebreak

% CinIME -----------------------------------------------------------------------
\begin{subsecao}{CinIME}

\figurapequenainline{cinime}

Convidamos vocês a conhecer o CinIME!! Somos um projeto do CAMAT que promove o
lazer e a integração de IMEanos, entre docentes, discentes e funcionários por
meio de sessões de cinema em que vocês escolhem o filme!

O CinIME ocorre toda sexta-feira, às 16h, na sala B5. A sessão, o refrigerante
e a pipoca são grátis. Funciona assim: você entra na sala, se serve do que
quiser, senta, relaxa e se diverte assistindo ao filme da semana ao lado dos
seus amigos.

A escolha do filme é feita através de votação no grupo “CinIME USP”, no
facebook. Para participar, primeiro você acessa e curte a página do CinIME no
facebook (https://www.facebook.com/cinimeusp) para ter acesso às nossas
atualizações. Então, na aba “Sobre”, você clica no link de sugestões apontado
na descrição da página. Dentre todas as sugestões de filmes, cinco são
escolhidos pela equipe do CinIME para a votação, priorizando a diversidade de
gêneros cinematográficos. Depois disso, a votação é lançada no grupo “CinIME
USP”, e o resultado é sempre anunciado ao fim da sessão da semana em curso, que ocorre
na sexta à tarde. O filme eleito é divulgado nos grupos
dos bixes e no grupo “IME-USP”, no facebook, e também por cartazes espalhados
pelo instituto. Uma vez por mês, a equipe do CinIME toma a liberdade de
escolher o filme da sessão sem votação.

Não se esqueçam de sugerir filmes, votar e comparecer ao CinIME!

Curtam nossa página no facebook e fiquem por dentro das novidades!

Parabéns e boa sorte nesta nova jornada, bixes!

\begin{description}
  \item[Facebook:] \url{https://www.facebook.com/CinIMEUSP}
\end{description}

\end{subsecao}


% Grupo A5 ---------------------------------------------------------------------
\begin{subsecao}{Grupo A5}

\figurapequenainline{grupo_A5}

Somos um grupo de estudantes da graduação e pós-graduação do IME e, anualmente,
organizamos eventos acadêmicos no instituto. Tudo começou em 2012, quando dois
alunos da pura perceberam que alguns assuntos muito interessantes sobre
matemática e sobre o meio acadêmico nem sempre eram desenvolvidos em sala de
aula, e resolveram organizar um evento para levar um destes assuntos aos demais
estudantes instituto. Assim, juntamente ao CAMat, organizaram um ciclo de
palestras sobre "Os 7 Problemas do Milênio", e o evento foi um sucesso.  Vários
professores e estudantes começaram a pedir que mais eventos como este fossem
organizados, e foi então que um deles teve a ideia de criar um grupo
independente das demais instituições do IME (sim, o Grupo A5 é independente. Não
é vinculado ao CAMat e nem a nenhum outro grupo, apesar de aceitar parcerias em
alguns eventos), com a finalidade de complementar a formação dos estudantes do
IME e de quem quiser participar, levando palestras e outros eventos sobre temas
relevantes e não explorados no currículo, de forma gratuita e com linguagem de
fácil entendimento.  E assim, no final de 2013, o Grupo A5 oficialmente nasceu,
com nome e logo e com novos integrantes no grupo.  Desde então, não paramos
mais. Em 2014 organizamos o ciclo de palestras "IC ou Não IC? - Eis a Questão",
em 2015 organizamos o evento "História da Matemática", que foi indicado como um
dos destaques de Cultura e Extensão de 2015 pelo IME. E em 2018 realizamos, em 
parceria com o Existimos, o ciclo de palestras "Mulheres no Mundo Corporativo",
que foi um sucesso! 

Para mais informações do Grupo A5 e dos eventos já organizados por nós, acessem
nosso site \url{www.ime.usp.br/~acinco} (ainda está em construção, mas em breve os
vídeos das palestras e demais informações estarão lá). E não deixem de curtir
a página Grupo A5 no facebook: \url{www.fb.com/pagina.GrupoA5} (é aqui que
vocês terão em primeira mão os detalhes de tudo o que for feito por nós).

Vale ressaltar que o Grupo A5 é formado e mantido por estudantes do IME, então o
sucesso e a continuidade do grupo dependem de todos; começando por vocês,
bixes. Então venham, assistam, dêem sugestões, participem. E se gostarem,
juntem-se ao nosso grupo!

\end{subsecao}


% Olimpíadas de Matemática e Informática ---------------------------------------
% Maratona de Programação ------------------------------------------------------
\begin{subsecao}{Olimpíadas de Conhecimento}

\begin{itemize}

\item{\bf Matemática: }

Bom pessoal, se vocês entraram no IME, muito provavelmente já participaram
de alguma Olimpíada de Matemática no Ensino Fundamental e/ou Médio. A
boa notícia é que vocês vão poder continuar participando se quiserem,
e quem nunca participou tem a oportunidade de começar agora.

Mas por que participar? As Olimpíadas Universitárias de Matemática são uma
oportunidade de se divertir resolvendo problemas difíceis de Matemática e agregar
valor ao currículo ao mesmo tempo. Elas são parecidas com as Olimpíadas de
Ensino Médio, mas com conteúdo de Matemática da graduação (essencialmente
Cálculo, Análise, Álgebra Linear, Álgebra, Combinatória e Teoria dos Números),
mas com enfoque em problemas que exigem criatividade e técnicas mais inovadoras,
muitas das quais vocês provavelmente não verão durante toda a graduação.

De quais olimpíadas podemos participar? Como alunos de graduação, vocês podem
participar da Olimpíada Iberoamericana de Matemática Universitária (OIMU),
Olimpíada Brasileira de Matemática (OBM) e Olimpíada Internacional de
Matemática (IMC).

Como fazemos para nos preparar? Os sites institucionais dessas olimpíadas
têm todo o material necessário para vocês que querem estudar e se preparar
para elas.

Como fazemos para participar? Inscrevam-se pelo site ou entrem em contato com
o professor Yoshiharu. Para o IMC aconselha-se ter ganhado medalha na OBM,
já que é necessário apoio financeiro do IME por ser uma olimpíada internacional.

%REFTIME
Mas nós, bixes, temos chance? Como foi o desempenho de IMEanos nelas? Nós
obtivemos sucesso nestas olimpíadas. Ganhamos medalhas em todas as três
competições e o resultado mais recente foi uma medalha de bronze na IMC e ouro
na OBM.

Se tiverem alguma dúvida, não hesitem em perguntar a algum veterane sobre os
Olímpicos!

Links institucionais:

\begin{description}
  \item[] \url{http://www.cimat.mx/Eventos/oimu/}
  \item[] \url{http://www.imc-math.org/}
  \item[] \url{http://www.obm.org.br/opencms/}
\end{description}

\item{\bf Informática: }

\textit{``Informática? Vocês mexem com Word, Excel e PowerPoint então?''}

Responder essa pergunta já virou rotina para competidores da Olimpíada
Brasileira de Informática (OBI). Não, Informática não é Word. Oras, então o que é a OBI?

A OBI é uma competição de lógica, matemática e computação. As provas envolvem
alguns problemas que você deve resolver com programas de computador.

Esta competição, na graduação, é exclusiva para ingressantes recém formados do
ensino médio. Quer dizer que vocês são a nossa única esperança de trazer mais
gloriosas medalhas ao IME! Isso também quer dizer que essa é a sua única chance
de participar da OBI, uma competição relativamente tranquila comparada à
Maratona de Programação.

Para participar, basta falar com o MaratonIME
(\url{https://www.ime.usp.br/~maratona/}), um grupo de extensão focado nesse
tipo de competição que promete te ajudar a se inscrever e se preparar, ou com o
Professor Carlinhos (\url{http://www.ime.usp.br/~cef/}).

Para mais informações, acessem \url{http://olimpiada.ic.unicamp.br/}.

\item{\bf Maratona de Programação: }

À primeira vista, a Maratona de Programação pode soar um tanto
surreal. Nerds correndo pela USP ao mesmo tempo que resolvem
problemas de computação e matemática? Infelizmente esse não
é o caso.

A Maratona de Programação se resume à resolução de problemas.
Se você adora resolver desafios, quebrar a cabeça com novos
e excitantes problemas e acumular toneladas de dinheiro, esse
é o lugar perfeito para você!

A competição consiste em uma série de problemas que englobam
temas como programação dinâmica, grafos e estruturas de dados.
Times de três pessoas devem resolver a maior quantidade de
desafios em cinco horas de programação. E tudo isso com direito
a um lanche gratuito durante a prova.

Mas não temam, bixes. Não é só por que vocês acabaram de entrar que
a probabilidade de ganhar uma medalha seja nula. Inclusive, na primeira
fase da maratona, uma equipe de bixes tem vaga garantida para a
fase brasileira.

Além da fama, constantes pedidos por autógrafos e dinheiro de sobra,
a maratona também vai lhes trazer um conhecimento muito mais
adiantado em relação ao dos seus colegas de classe, e até oportunidades
de emprego em empresas de renome, como Google, Facebook e IBM.

Se vocês se interessaram pela maratona e querem saber os horários dos
treinos, como participar ou saber mais, acessem:

\begin{description}
  \item[Site:] \url{http://www.ime.usp.br/~maratona}
  \item[Site da competição nacional:] \url{http://maratona.ime.usp.br/}
\end{description}

\end{itemize}


\end{subsecao}


%FIXME GAMBIARRA para não quebrar página num lugar zuado
\pagebreak

% Fala Sério -------------------------------------------------------------------
\begin{subsecao}{Fala Sério}

\figurapequenainline{fala-serio}

A iniciativa Fala Sério é um grupo de extensão de cunho social da USP que atua em
escolas públicas de São Paulo proporcionando reflexões sobre futuro e felicidade.
O projeto se iniciou na Escola Politécnica da USP e devido ao seu grande sucesso 
se tornou um projeto de extensão da USP.

Para muitos alunos o processo de transição do Ensino Médio para a vida adulta é 
muito difícil e muitas vezes perturbador. Surgem muitas dúvidas, insegurança, 
cobranças; e é neste contexto que o Fala Sério atua.

Por meio de dois encontros levamos reflexões e espaços de conversa sobre 
felicidade e futuro; quais nossos sonhos e como alcançá-los. Desse modo, 
oferecemos um apoio pessoal que muitas vezes os jovens não encontram em suas 
escolas e que faz uma grande diferença em suas vidas.

Eaí, se animou com a ideia? Quer compartilhar suas experiências e entrar nessa 
missão com a gente? Visite nossa página no facebook e vem nos conhecer:
\url{facebook.com/iniciativafalaserio}

\end{subsecao}


\end{secao}
